% Options for packages loaded elsewhere
% Options for packages loaded elsewhere
\PassOptionsToPackage{unicode}{hyperref}
\PassOptionsToPackage{hyphens}{url}
\PassOptionsToPackage{dvipsnames,svgnames,x11names}{xcolor}
%
\documentclass[
  letterpaper,
  DIV=11,
  numbers=noendperiod]{scrartcl}
\usepackage{xcolor}
\usepackage{amsmath,amssymb}
\setcounter{secnumdepth}{-\maxdimen} % remove section numbering
\usepackage{iftex}
\ifPDFTeX
  \usepackage[T1]{fontenc}
  \usepackage[utf8]{inputenc}
  \usepackage{textcomp} % provide euro and other symbols
\else % if luatex or xetex
  \usepackage{unicode-math} % this also loads fontspec
  \defaultfontfeatures{Scale=MatchLowercase}
  \defaultfontfeatures[\rmfamily]{Ligatures=TeX,Scale=1}
\fi
\usepackage{lmodern}
\ifPDFTeX\else
  % xetex/luatex font selection
\fi
% Use upquote if available, for straight quotes in verbatim environments
\IfFileExists{upquote.sty}{\usepackage{upquote}}{}
\IfFileExists{microtype.sty}{% use microtype if available
  \usepackage[]{microtype}
  \UseMicrotypeSet[protrusion]{basicmath} % disable protrusion for tt fonts
}{}
\makeatletter
\@ifundefined{KOMAClassName}{% if non-KOMA class
  \IfFileExists{parskip.sty}{%
    \usepackage{parskip}
  }{% else
    \setlength{\parindent}{0pt}
    \setlength{\parskip}{6pt plus 2pt minus 1pt}}
}{% if KOMA class
  \KOMAoptions{parskip=half}}
\makeatother
% Make \paragraph and \subparagraph free-standing
\makeatletter
\ifx\paragraph\undefined\else
  \let\oldparagraph\paragraph
  \renewcommand{\paragraph}{
    \@ifstar
      \xxxParagraphStar
      \xxxParagraphNoStar
  }
  \newcommand{\xxxParagraphStar}[1]{\oldparagraph*{#1}\mbox{}}
  \newcommand{\xxxParagraphNoStar}[1]{\oldparagraph{#1}\mbox{}}
\fi
\ifx\subparagraph\undefined\else
  \let\oldsubparagraph\subparagraph
  \renewcommand{\subparagraph}{
    \@ifstar
      \xxxSubParagraphStar
      \xxxSubParagraphNoStar
  }
  \newcommand{\xxxSubParagraphStar}[1]{\oldsubparagraph*{#1}\mbox{}}
  \newcommand{\xxxSubParagraphNoStar}[1]{\oldsubparagraph{#1}\mbox{}}
\fi
\makeatother


\usepackage{longtable,booktabs,array}
\usepackage{calc} % for calculating minipage widths
% Correct order of tables after \paragraph or \subparagraph
\usepackage{etoolbox}
\makeatletter
\patchcmd\longtable{\par}{\if@noskipsec\mbox{}\fi\par}{}{}
\makeatother
% Allow footnotes in longtable head/foot
\IfFileExists{footnotehyper.sty}{\usepackage{footnotehyper}}{\usepackage{footnote}}
\makesavenoteenv{longtable}
\usepackage{graphicx}
\makeatletter
\newsavebox\pandoc@box
\newcommand*\pandocbounded[1]{% scales image to fit in text height/width
  \sbox\pandoc@box{#1}%
  \Gscale@div\@tempa{\textheight}{\dimexpr\ht\pandoc@box+\dp\pandoc@box\relax}%
  \Gscale@div\@tempb{\linewidth}{\wd\pandoc@box}%
  \ifdim\@tempb\p@<\@tempa\p@\let\@tempa\@tempb\fi% select the smaller of both
  \ifdim\@tempa\p@<\p@\scalebox{\@tempa}{\usebox\pandoc@box}%
  \else\usebox{\pandoc@box}%
  \fi%
}
% Set default figure placement to htbp
\def\fps@figure{htbp}
\makeatother





\setlength{\emergencystretch}{3em} % prevent overfull lines

\providecommand{\tightlist}{%
  \setlength{\itemsep}{0pt}\setlength{\parskip}{0pt}}



 


\KOMAoption{captions}{tableheading}
\makeatletter
\@ifpackageloaded{caption}{}{\usepackage{caption}}
\AtBeginDocument{%
\ifdefined\contentsname
  \renewcommand*\contentsname{Table of contents}
\else
  \newcommand\contentsname{Table of contents}
\fi
\ifdefined\listfigurename
  \renewcommand*\listfigurename{List of Figures}
\else
  \newcommand\listfigurename{List of Figures}
\fi
\ifdefined\listtablename
  \renewcommand*\listtablename{List of Tables}
\else
  \newcommand\listtablename{List of Tables}
\fi
\ifdefined\figurename
  \renewcommand*\figurename{Figure}
\else
  \newcommand\figurename{Figure}
\fi
\ifdefined\tablename
  \renewcommand*\tablename{Table}
\else
  \newcommand\tablename{Table}
\fi
}
\@ifpackageloaded{float}{}{\usepackage{float}}
\floatstyle{ruled}
\@ifundefined{c@chapter}{\newfloat{codelisting}{h}{lop}}{\newfloat{codelisting}{h}{lop}[chapter]}
\floatname{codelisting}{Listing}
\newcommand*\listoflistings{\listof{codelisting}{List of Listings}}
\makeatother
\makeatletter
\makeatother
\makeatletter
\@ifpackageloaded{caption}{}{\usepackage{caption}}
\@ifpackageloaded{subcaption}{}{\usepackage{subcaption}}
\makeatother
\usepackage{bookmark}
\IfFileExists{xurl.sty}{\usepackage{xurl}}{} % add URL line breaks if available
\urlstyle{same}
\hypersetup{
  pdftitle={Syllabus},
  colorlinks=true,
  linkcolor={blue},
  filecolor={Maroon},
  citecolor={Blue},
  urlcolor={Blue},
  pdfcreator={LaTeX via pandoc}}


\title{Syllabus}
\author{}
\date{}
\begin{document}
\maketitle


\subsection{Course Information}\label{course-information}

\begin{itemize}
\tightlist
\item
  \textbf{Course Name}: Data Manipulation and Visualization
\item
  \textbf{Course Code}: CSCI-2025
\end{itemize}

\subsubsection{Class meetings}\label{class-meetings}

\begin{longtable}[]{@{}
  >{\raggedright\arraybackslash}p{(\linewidth - 4\tabcolsep) * \real{0.1972}}
  >{\raggedright\arraybackslash}p{(\linewidth - 4\tabcolsep) * \real{0.3944}}
  >{\raggedright\arraybackslash}p{(\linewidth - 4\tabcolsep) * \real{0.4085}}@{}}
\toprule\noalign{}
\begin{minipage}[b]{\linewidth}\raggedright
Meeting
\end{minipage} & \begin{minipage}[b]{\linewidth}\raggedright
Location
\end{minipage} & \begin{minipage}[b]{\linewidth}\raggedright
Time
\end{minipage} \\
\midrule\noalign{}
\endhead
\bottomrule\noalign{}
\endlastfoot
Lecture & Cruzen-Murray Library, 208 & Mon-Thurs 1 - 3 pm \\
Office Hours & Boone 126B & Mon-Thurs 10 - 11 am \\
\end{longtable}

Office hours are also available by appointment, just email me!

\subsubsection{Instructor Information}\label{instructor-information}

\begin{itemize}
\tightlist
\item
  \textbf{Instructor}: Professor Eric Friedlander
\item
  \textbf{Office}: Boone 126B
\item
  \textbf{Email}:
  \href{mailto:efriedlander@collegeofidaho.edu}{\nolinkurl{efriedlander@collegeofidaho.edu}}
\end{itemize}

\subsubsection{Catalog Description}\label{catalog-description}

Methods for cleaning and manipulating data, including grouping,
split-apply-combine, and moving between long and wide formats, will be
covered. These techniques will be combined with visualization techniques
to produce readable, informative graphics of complex, high-dimensional
datasets. Formerly CSC-285.

\subsubsection{Pre-requisites}\label{pre-requisites}

MATH-2025 or CSCI-1040, grade of C or better

\subsection{Course Learning
Objectives}\label{course-learning-objectives}

By the end of the semester, you will be able to\ldots{}

\begin{itemize}
\tightlist
\item
  Manipulate and clean data using R and the tidyverse.
\item
  Create informative visualizations of complex datasets using ggplot2.
\item
  Understand the principles of designing and creating effective data
  visualizations.
\item
  Communicate data analysis results effectively through written reports,
  dashboards, and presentations.
\end{itemize}

\subsection{Course community}\label{course-community}

\subsubsection{Communication}\label{communication}

All lecture notes, assignment instructions, an up-to-date schedule, and
other course materials may be found on the course website,
\href{https://csci2025wi26.netlify.app}{csci2025wi26.netlify.app}.

Periodic announcements will be sent via email and will also be available
through Canvas and grades will be stored in the Canvas gradebook. Please
check your email regularly to ensure you have the latest announcements
for the course.

\subsubsection{In class agreements}\label{in-class-agreements}

If we discuss/agree to something in class or office hours which requires
action from me (e.g.~``you may turn in your homework late due to a
sporting event''), you MUST send me a follow-up message. If you don't, I
will almost certainly forget, and our agreement will be considered null
and void.

\subsubsection{Getting help in the
course}\label{getting-help-in-the-course}

\begin{itemize}
\tightlist
\item
  If you have a question during lecture, feel free to ask it! There are
  likely other students with the same question, so by asking you will
  create a learning opportunity for everyone.
\item
  I am here to help you be successful in the course. You are encouraged
  to attend \emph{office hours} to ask questions about the course
  content and assignments. Many questions are most effectively answered
  as you discuss them with others, so office hours are a valuable
  resource. You are encouraged to use them!
\end{itemize}

\subsubsection{Email}\label{email}

If you have questions about assignment extensions or accommodations,
please email
\href{mailto:efriedlander@collegeofidaho.edu}{\nolinkurl{efriedlander@collegeofidaho.edu}}.
Please see \hyperref[late-work-policy]{Late work policy} for more
information. Barring extenuating circumstances, I will respond to CSCI
2025 emails within 48 hours Monday - Friday. Response time may be slower
for emails sent Friday evening - Sunday.

Check out the \href{support.qmd}{Support} page for more resources.

\subsection{Textbook}\label{textbook}

There is no official textbook for this course. However, readings may be
assigned from the following texts (all freely available online).

\begin{itemize}
\tightlist
\item
  \href{https://r4ds.had.co.nz/}{R for Data Science} by Garret Grolemund
  and Hadley Wickham
\end{itemize}

\subsection{Lectures}\label{lectures}

This will be a flipped classroom. Before each lecture, you will be
assigned readings and videos to prepare you for the material we will
cover in class. During class time, we will focus on applying the
concepts you learned in the readings through a variety of activities,
including coding exercises, discussions, and group work.

You are expected to bring a laptop to each class so that you can
participate in the in-class exercises. Please make sure your device is
fully charged before you come to class, as the number of outlets in the
classroom may not be sufficient to accommodate everyone.

\subsection{Activities \& Assessment}\label{activities-assessment}

You will be assessed based on four components: participation and
self-evluation, quizzes, a class project, and an indivdual project.

This course will be largely ``ungraded''. That is, you will be
completing many low-stakes activities that will not be graded. Instead,
you will self-assess what you have learned and how well you understand
the material. The purpose of this approach is to help you focus on
learning the material rather than on earning a grade.

At the end of the semester, Dr.~Friedlander will first determine whether
you deserve a passing grade. This will be largely be determine by
looking at your quiz scores and projects. If you pass this hurdle, then
your final grade will be determined based on your own assessment of the
overall quality of your work throughout the semester.

\subsubsection{Participation and
Self-Evaluation}\label{participation-and-self-evaluation}

After each class period, you will be expected to complete a brief
self-evaluation reflecting on your understanding of the material covered
in class. These self-evaluations will help you identify areas where you
need to improve and will also help me understand how well the class is
grasping the material. After each class period you will be asked to
response to the following questions:

\begin{enumerate}
\def\labelenumi{\arabic{enumi}.}
\tightlist
\item
  How well do I understand the material covered in lectures and readings
  for today's class?
\item
  Did I put in a good faith effort to prepare for and participate in
  today's class?
\item
  How would I rate my level of engagement during today's class?
\item
  How well did I perform on the in-class activities?
\item
  How did I learn and grow from the readings, lectures, and in-class
  activities today?
\item
  What overall grade would I give myself for the work I completed since
  the end of the last class period?
\end{enumerate}

\subsubsection{Group Project}\label{group-project}

In the group project the course will work together to develop a data
dashboard for an external client. More information on this to come.

\subsubsection{Individual Project}\label{individual-project}

The purpose of the individual project is to apply what you've learned
throughout the semester to analyze an interesting data-driven research
question or to tell an interesting story using data. The project will be
completed individually, and each student will present their work through
a written report and in-class presentation. More information about the
project will be provided during the semester.

\subsubsection{Quizzes}\label{quizzes}

Dr.~Friedlander will administer short quizzes periodically throughout
the semester to assess your understanding of the course material. These
quizzes will take place at the beginning of class, will not be announced
in advance, and will cover the readings and lectures. The purpose of
these quizzes is to encourage you to keep up with the course material
and to provide me with feedback on how well the class is grasping the
material and to ensure that all students are coming to class ready to
participate. If you are late to class or absent on a quiz day, you will
not be able to make up the quiz unless you notify me beforehand that you
will be missing class for a valid reason (e.g., illness, family
emergency, school-sponsored event).

\textbf{At the end of the semester, if you receive less than a 60\%
average on the quizzes, you will not pass the course regardless of your
performance on other assignments.}

\subsection{Way's to fail this course}\label{ways-to-fail-this-course}

\begin{itemize}
\tightlist
\item
  Receiving less than a 60\% average on the quizzes.
\item
  Failing to complete either of the projects.
\item
  Missing more than 3 class period without a valid excuse.
\item
  Failing to turn in more than 3 self-evaluations.
\end{itemize}

\subsection{Five tips for success}\label{five-tips-for-success}

Your success in this course depends very much on you and the effort you
put into it. The course has been organized so that the burden of
learning is on you. I will help you by providing you with materials and
answering questions and setting a pace, but for this to work you must do
the following:

\begin{enumerate}
\def\labelenumi{\arabic{enumi}.}
\item
  Complete all the preparation work before class.
\item
  Ask questions. As often as you can. In class, out of class. Ask me,
  ask your friends, ask the person sitting next to you. This will help
  you more than anything else. If you get a question wrong on an
  assessment, ask why. If you're not sure about the homework, ask. If
  you hear something on the news that sounds related to what we
  discussed, ask. If the reading is confusing, ask.
\item
  Do the readings and work outside of class.
\item
  Don't procrastinate. The content builds upon what was taught in
  previous weeks, so if something is confusing to you on Day 2, Day 3
  will become more confusing, Day 4 even worse, etc. Don't let the week
  end with unanswered questions. But if you find yourself falling behind
  and not knowing where to begin asking, come to office hours and I can
  help you identify a good (re)starting point.
\end{enumerate}

\subsection{Course policies}\label{course-policies}

\subsubsection{Academic honesty}\label{academic-honesty}

\textbf{TL;DR: Don't cheat!}

\begin{itemize}
\item
  For the projects, collaboration within teams is not only allowed, but
  expected. Communication between teams at a high level is also allowed
  however you may not share code or components of the project across
  teams.
\item
  I typically have a very long section about academic honesty in my
  syllabi. However, since this course is largely ungraded, I will simply
  say that I expect you to adhere to the College of Idaho Honor Code in
  all your work for this course. If you have any questions about what
  constitutes academic dishonesty in this course, please ask me.
\item
  \textbf{You are responsible of everything you turn in.} For the most
  part, you are welcome to use any resources you like to help you
  complete assignments. However, you are responsible for ensuring that
  you can not only understand, but explain everything that you turn in.
  If you can't, you will not receive credit for the assignment.
\end{itemize}

\subsubsection{Late work policy}\label{late-work-policy}

The due dates for projects are there to help you keep up with the course
material and to ensure I can provide feedback within a timely manner. I
understand that things come up periodically that could make it difficult
to submit an assignment by the deadline.

\begin{itemize}
\tightlist
\item
  \textbf{School-Sponsored Events/Illness:} If a deadline must be missed
  due to a school-sponsored event, you must let me know at least a week
  ahead of time so that we can schedule a time for you to make up the
  work before you leave. If you must miss a exam or a project
  presentation due to illness, you must let me know before class that
  day so that we can schedule a time for you to take a make-up quiz or
  exam. Failure to adhere to this policy will result in a 35\% penalty
  on the corresponding assignment.
\end{itemize}

\subsubsection{College of Idaho Honor
Code}\label{college-of-idaho-honor-code}

\begin{quote}
The College of Idaho maintains that academic honesty and integrity are
essential values in the educational process. Operating under an Honor
Code philosophy, the College expects conduct rooted in honesty,
integrity, and understanding, allowing members of a diverse student body
to live together and interact and learn from one another in ways that
protect both personal freedom and community standards. Violations of
academic honesty are addressed primarily by the instructor and may be
referred to the
\href{https://collegeofidaho.smartcatalogiq.com/en/current/Undergraduate-Catalog/Policies-and-Procedures/Academic-Misconduct}{Student
Judicial Board}.
\end{quote}

By participating in this course, you are agreeing that all your work and
conduct will be in accordance with the College of Idaho Honor Code.

\subsubsection{Disability Accommodation
Statement}\label{disability-accommodation-statement}

The College of Idaho seeks to provide an educational environment that is
accessible to the needs of students with disabilities. The College
provides reasonable services to enrolled students who have a documented
permanent or temporary physical, psychological, learning, intellectual,
or sensory disability that qualifies the student for accommodations
under the Americans with Disabilities Act or section 504 of the
Rehabilitation Act of 1973. If you have, or think you may have, a
disability that impacts your performance as a student in this class, you
are encouraged to arrange support services and/or accommodations through
the Department of Accessibility and Learning Excellence located in
McCain 201B and available via email at
\href{mailto:accessibility@collegeofidaho.edu}{\nolinkurl{accessibility@collegeofidaho.edu}}.
Reasonable academic accommodations may be provided to students who
submit appropriate and current documentation of their disability.
Accommodations can be arranged only through this process and are not
retroactively applied. More information can be found on the DALE webpage
(\url{https://www.collegeofidaho.edu/accessibility}).




\end{document}
